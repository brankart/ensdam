\documentclass[11pt]{article}

\usepackage[latin1]{inputenc}
\usepackage[a4paper,top=1in,left=1in,right=1in,bottom=1in]{geometry}
\usepackage[pdftex]{graphicx}
\usepackage{natbib}

\begin{document}

\pagestyle{empty}

\centerline{
\includegraphics[height=22mm]{Logos/logo_uga.png}
\hspace{5mm}
\includegraphics[height=22mm]{Logos/logo_cnrs.png}
\hfill
\includegraphics[height=22mm]{Logos/logo_ige.png}
}

\vspace{20mm}

\begin{center}

{\Huge\bf StochTools}

\vspace{10mm}

{\Large\bf Stochastic Modelling Tools}

\vspace{10mm}

{\Large\bf User's guide}

\vspace{10mm}

{\large\bf Jean-Michel Brankart}

\vspace{5mm}
{\tt http://pp.ige-grenoble.fr/pageperso/brankarj/}

\vspace{5mm}
{\large Institut des G\'eosciences de l'Environnement}

\vspace{1mm}
{\large Universit\'e Grenoble Alpes, CNRS, France}

\end{center}

\vspace{20mm}
The main purpose of StochTools is to provide tools
to generate random numbers, random fields,
or stochastic processes of various types,
to be used in stochastic modelling or data assimilation systems.

The tools are provided as a library of modules,
which can be easily plugged in any existing software.
This library includes:

\begin{itemize}
\item the generation of random numbers with various probability distribution
      (uniform, normal, gamma, beta, exponential, truncated exponential, truncated normal);
\item the computation of the probabilty density function (pdf), the cumulative distribution function (cdf)
      or the inverse cdf of several probability distributions (normal, gamma, beta);
\item the computation of the cdf of the product of two normal random numbers;
\item the transformation of a normal random variable into a gamma or a beta random variable,
      and the transformation of the product of two normal random variables into a normal variable;
\item the generation of random fields with specified spectrum (in 1D, in 2D or
      in the basis of the spherical harmonics).
\end{itemize}

\clearpage

\pagestyle{plain}

\section{Description of the modules}

In this section,
the modules are described one by one,
giving for each of them:
the method that has been implemented,
the list of public variables and public routines
(with a description of input and output data),
the MPI parallelization, and
an estimation of the computational cost
as a function of the size of the problem.

\subsection{Module: {\tt\bf storng}}

The purpose of this module is to generate random numbers,
according to a few specific distribution:
uniform, Gaussian, gamma, and beta.

\subsubsection*{Method}

The module is based on (and includes) the
64-bit KISS (Keep It Simple Stupid) random number generator
distributed by George Marsaglia.
KISS is based on 3 components:
(1) Xorshift (XSH), period $2^{64}-1$,
(2) Multiply-with-carry (MWC), period ($2^{121}+2^{63}-1$)
(3) Congruential generator (CNG), period $2^{64}$.
The overall period of the sequence of radom numbers is:
$(2^{250}+2^{192}+2^{64}-2^{186}-2^{129})/6 \simeq 2^{247.42} \,\mbox{or}\, 10^{74.48}$.

\subsubsection*{Public routines}

\begin{description}
\item[kiss:] KISS random number generator (64-byte integers);
\item[kiss\_seed:] define seeds for KISS random number generator;
\item[kiss\_save:] save current state of KISS (for future restart);
\item[kiss\_load:] load the saved state of KISS;
\item[kiss\_reset:] reset to the default seeds;
\item[kiss\_check:] check the KISS pseudo-random sequence;
\item[kiss\_uniform:] real random numbers with uniform distribution in [0,1];
\item[kiss\_gaussian:] real random numbers with Gaussian distribution N(0,1);
\item[kiss\_gamma:] real random numbers with Gamma distribution Gamma(k,1);
\item[kiss\_beta:] real random numbers with Beta distribution Beta(a,b).
\end{description}

\subsubsection*{Computational cost}

This is a very cheap random number generator.
Each call to the {\tt kiss} function requires
the following list of operations
applied to 64-byte integers:

\centerline{* (1), + (6), ISHFT (4), IEOR (3), .EQ. (1)}

\subsection{Module: {\tt\bf stotge}}

The purpose of this module is to generate random numbers
with truncated normal or truncated exponential distribution.

\subsubsection*{Public routines}

\begin{description}
\item[ran\_te:] sample random number with truncated exponential distribution;
\item[ran\_tg:] sample random number with truncated Gaussian distribution;
\item[ranv\_tg:] sample random vector with truncated Gaussian distribution.
\end{description}

\subsection{Module: {\tt\bf stoutil}}

The purpose of this module is to compute the probabilty density function (pdf),
the cumulative distribution function (cdf) or the inverse cdf
of several probability distributions (normal, gamma, beta).

\subsubsection*{Public routines}

\begin{description}
\item[pdf\_gaussian:] compute Gaussian pdf;
\item[logpdf\_gaussian:] compute the logarithm of a Gaussian pdf (minus a constant);
\item[cdf\_gaussian:] compute Gaussian cdf;
\item[invcdf\_gaussian:] compute Gaussian inverse cdf;
\item[pdf\_gamma:] compute gamma pdf;
\item[logpdf\_gamma:] compute the logarithm of a gamma pdf (minus a constant);
\item[cdf\_gamma:] compute gamma cdf;
\item[invcdf\_gamma:] compute gamma inverse cdf;
\item[pdf\_beta:] compute beta pdf;
\item[logpdf\_beta:] compute the logarithm of a beta pdf (minus a constant);
\item[cdf\_beta:] compute beta cdf;
\item[invcdf\_beta:] compute beta inverse cdf.
\end{description}

\subsection{Module: {\tt\bf stogprod}}

The purpose of this module is to compute the cumulative distribution function (cdf)
of the product of two normal random variables.
This module is based on Alan Miller'implementation of the cdf
({\tt https://jblevins.org/mirror/amiller/}).

\subsubsection*{Public routines}

\begin{description}
\item[fnprod:] compute the cdf of the product of two normal random variables.
\end{description}

\subsection{Module: {\tt\bf stoanam}}

The purpose of this module is to transform a random variable
from one distribution to another.

\subsubsection*{Public routines}

\begin{description}
\item[gau\_to\_gam:] transform a normal random variable into a gamma random variable;
\item[gau\_to\_beta:] transform a normal random variable into a beta random variable;
\item[gprod\_to\_gau:] transform the product of two normal random variables
                       into a normal random variable.
\end{description}

\subsection{Module: {\tt\bf storfg}}

The purpose of this module is to generate random fields with specified spectrum
in the basis of the spherical harmonics.
Routines to generate 1D or 2D random fields with a continuous spectrum are also included
in the module, but they need to be reconsidered and they are not described below.

\subsubsection*{Method}

The approach is to generate two-dimensional random fields $w(\theta,\varphi)$,
function of latitude ($\theta$) and longitude ($\varphi$),
by linear combination of the sperical harmonics $Y_{lm}(\theta,\varphi)$:

\begin{equation}
\label{eq:ranw}
w(\theta,\varphi) = \sum_{l=0}^{l_{\max}} \sum_{m=-l}^l
                    a_{lm} \, \xi_{lm} \, Y_{lm}(\theta,\varphi)
\end{equation}

\noindent
where $l$ and $m$ are the degree and order of each spherical harmonics,
$\xi_{lm}$ are independent Gaussian noises (with zero mean and unit variance),
and $a_{lm}$ are spectral amplitudes defining the spatial
correlation structure of the random field.

The amplitudes~$a_{lm}$ defines the spectrum of~$w$
in the basis of spherical harmonics, for instance:

\begin{equation}
\label{eq:amplw}
a^2_{lm} = \frac{k}{2l+1} \left[ 1 + \left( l/l_c \right)^{2p} \right]^{-1}
\end{equation}

\noindent
where the degree~$l_c$ defines the characteristic length scale
($\ell_c=R_c/l_c$, where $R_c$ is the earth radius),
the exponent~$p$ modifies the shape of the spectrum, and
the coefficient~$k$ is chosen to specify the variance ($\sigma^2$) of~$w$,
i.e.\ so that:

\begin{equation}
\label{eq:amplw-norm}
\sum_{l=0}^{l_{\max}} \sum_{m=-l}^l a^2_{lm} = \sigma^2
\end{equation}

\noindent
Choosing the~$a_{lm}$ independent of~$m$ [as in Eq.~(\ref{eq:amplw})]
means that the random field~$w$ is homogeneous and isotropic on the sphere.

\subsubsection*{Public routines}

\begin{description}
\item[gen\_field\_2s:] to generate the random field.
  \begin{description}
  \item[{\tt ranfield} (output)]: random field on the requested grid;
  \item[{\tt lon} (input)]: longitude of grid points;
  \item[{\tt lat} (input)]: latitude of grid points;
  \item[{\tt pow\_spect} (input)]: callback routine providing the requested power spectrum;
  \item[{\tt lmin} (input)]: minimum degree of the spherical harmonics;
  \item[{\tt lmax} (input)]: maximum degree of the spherical harmonics.
  \end{description}
\end{description}

\subsubsection*{Computational cost}

The computational complexity (leading asymptotic behaviour for large systems)
is given by:

\begin{equation}
C \sim k n  l_{\max}^2 = k n \left( \frac{2\pi R}{\lambda_c} \right)^2
\end{equation}

\noindent
where $n$ is the number of grid points in the random field,
$l_{\max}$ is the maximum degree of spherical harmonics, and
$k$ is the number of operation required for one single evaulation
of spherical harmonics.

\section{Examples}

In this section,
the examples provided with the library are described,
giving for each of them:
the purpose of the examples,
the list of modules/routines that are illustrated,
the input parameters and data,
the calling sequence of the library routines,
with a description of inputs and outputs for each of them, and
a description of the final result that is expected.

\subsection{Random field on the sphere}

The purpose of this example is to illustrate
the generation of a random field on the sphere.
Input data are: the maximum degree of the spherical harmonics
used to generate the random field, the power spectrum of the random field,
and the definition of the output grid
(where to provide the random field).
The output is a random field on the sphere written in NetCDF.


\end{document}

