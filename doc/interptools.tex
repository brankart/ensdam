\documentclass[11pt]{article}

\usepackage[latin1]{inputenc}
\usepackage[a4paper,top=1in,left=1in,right=1in,bottom=1in]{geometry}
\usepackage[pdftex]{graphicx}
\usepackage{natbib}

\begin{document}

\pagestyle{empty}

\centerline{
\includegraphics[height=22mm]{Logos/logo_uga.png}
\hspace{5mm}
\includegraphics[height=22mm]{Logos/logo_cnrs.png}
\hfill
\includegraphics[height=22mm]{Logos/logo_ige.png}
}

\vspace{20mm}

\begin{center}

{\Huge\bf InterpTools}

\vspace{10mm}

{\Large\bf Interpolation Tools}

\vspace{10mm}

{\Large\bf User's guide}

\vspace{10mm}

{\large\bf Jean-Michel Brankart}

\vspace{5mm}
{\tt http://pp.ige-grenoble.fr/pageperso/brankarj/}

\vspace{5mm}
{\large Institut des G\'eosciences de l'Environnement}

\vspace{1mm}
{\large Universit\'e Grenoble Alpes, CNRS, France}

\end{center}

\vspace{20mm}
The purpose of InterpTools is to provide tools
to localize data in grids and compute interpolation coefficients.

The tools are provided as a library of modules,
which can be easily plugged in any existing software.
This library includes:

\begin{itemize}
\item localization and interpolation in 1D grid;
\item localization and interpolation in 2D grid in Cartesian coordinates;
\item localization and interpolation in 2D grid on the sphere;
\end{itemize}

\clearpage

\pagestyle{plain}

\section{Description of the modules}

In this section,
the modules are described one by one,
giving for each of them:
the method that has been implemented,
the list of public variables and public routines
(with a description of input and output data),
the MPI parallelization, and
an estimation of the computational cost
as a function of the size of the problem.

\subsection{Module: {\tt\bf interp}}

The purpose of this module is to localize data in input grids
and compute linear interpolation coefficients.

\subsubsection*{Method}

\subsubsection*{Public variables}

None.

\subsubsection*{Public routines}

\begin{description}
\item[grid1D\_locate:] localize data in a 1D grid;
\item[grid1D\_interp:] compute interpolation coefficients in a 1D grid;
\item[grid2D\_init:] define the type of grid;
\item[grid2D\_locate:] localize data in a 2D grid (in cartesian or spherical coordinates);
\item[grid2D\_interp:] compute interpolation coefficients in a 2D grid (in cartesian or spherical coordinates);
\end{description}

\subsubsection*{MPI parallelization}

\subsubsection*{Computational cost}

\end{document}

