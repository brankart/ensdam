\documentclass[11pt]{article}

\usepackage[latin1]{inputenc}
\usepackage[a4paper,top=1in,left=1in,right=1in,bottom=1in]{geometry}
\usepackage[pdftex]{graphicx}
\usepackage{natbib}

\begin{document}

\pagestyle{empty}

\centerline{
\includegraphics[height=22mm]{Logos/logo_uga.png}
\hspace{5mm}
\includegraphics[height=22mm]{Logos/logo_cnrs.png}
\hfill
\includegraphics[height=22mm]{Logos/logo_ige.png}
}

\vspace{20mm}

\begin{center}

{\Huge\bf EnsStat}

\vspace{10mm}

{\Large\bf Ensemble Statistics}

\vspace{10mm}

{\Large\bf User's guide}

\vspace{10mm}

{\large\bf Jean-Michel Brankart}

\vspace{5mm}
{\tt http://pp.ige-grenoble.fr/pageperso/brankarj/}

\vspace{5mm}
{\large Institut des G\'eosciences de l'Environnement}

\vspace{1mm}
{\large Universit\'e Grenoble Alpes, CNRS, France}

\end{center}

\vspace{20mm}
The purpose of EnsStat is to provide tools
to compute basic ensemble statistics.

The tools are provided as a library of modules,
which can be easily plugged in any existing software.
This library includes:

\begin{itemize}
\item the computation of ensemble mean and standard deviation;
\item the computation of ensemble covariance or correlation structure.
\end{itemize}

\clearpage

\pagestyle{plain}

\section{Description of the modules}

In this section,
the modules are described one by one,
giving for each of them:
the method that has been implemented,
the list of public variables and public routines
(with a description of input and output data),
the MPI parallelization, and
an estimation of the computational cost
as a function of the size of the problem.

\subsection{Module: {\tt\bf meanstd}}

The purpose of this module is to compute
ensemble mean and standard deviation.

\subsubsection*{Method}

\subsubsection*{Public variables}

None.

\subsubsection*{Public routines}

\begin{description}
\item[ensemble\_meanstd:] compute ensemble mean and standard deviation.
  \begin{description}
  \item[{\tt ens} (input)]: input ensemble;
  \item[{\tt mean} (output)]: ensemble mean;
  \item[{\tt std} (output, optional)]: ensemble standard deviation;
  \item[{\tt weight} (input, optional)]: weight associated to each ensemble member
                                         (default=equal weights).
  \end{description}
\item[update\_meanstd:] update ensemble mean and ensemble mean square anomaly with new member.
  \begin{description}
  \item[{\tt vct} (input)]: additional ensemble member;
  \item[{\tt idx} (input, optional)]: index of new member
                                      (present if weight and weightsum are not present);
  \item[{\tt weight} (input, optional)]: weight associated to each ensemble member
                                         (default=equal weights).
  \item[{\tt weightsum} (input/output, optional)]: current sum of weights;
  \item[{\tt mean} (input/output)]: ensemble mean;
  \item[{\tt msqra} (input/output, optional)]: ensemble mean square anomaly.
  \end{description}
\end{description}

\subsubsection*{MPI parallelization}

\subsubsection*{Computational cost}

\subsection{Module: {\tt\bf meancov}}

The purpose of this module is to compute
ensemble covariance, ensemble correlation structure, or representers.

\subsubsection*{Method}

\subsubsection*{Public variables}

\begin{description}
\item[correlation\_missing\_value] missing value to use where no valid data is available (default=-9999.).
\end{description}

\subsubsection*{Public routines}

\begin{description}
\item[ensemble\_covariance:] compute ensemble covariance with respect to a scalar reference ensemble:
  \begin{description}
  \item[{\tt ens} (input)]: input ensemble;
  \item[{\tt ensref} (input)]: scalar reference ensemble;
  \item[{\tt cov} (output)]: ensemble covariance;
  \item[{\tt weight} (input, optional)]: weight associated to each ensemble member
                                         (default=equal weights).
  \end{description}
\item[ensemble\_correlation:] compute ensemble correlation with respect to a scalar reference ensemble:
  \begin{description}
  \item[{\tt ens} (input)]: input ensemble;
  \item[{\tt ensref} (input)]: scalar reference ensemble;
  \item[{\tt correl} (output)]: ensemble covariance;
  \item[{\tt weight} (input, optional)]: weight associated to each ensemble member
                                         (default=equal weights).
  \end{description}
\item[ensemble\_representer:] compute ensemble representer with respect to a scalar reference ensemble:
  \begin{description}
  \item[{\tt ens} (input)]: input ensemble;
  \item[{\tt ensref} (input)]: scalar reference ensemble;
  \item[{\tt representer} (output)]: ensemble covariance;
  \item[{\tt weight} (input, optional)]: weight associated to each ensemble member
                                         (default=equal weights).
  \end{description}
\item[update\_meancov:] update ensemble mean and ensemble mean product of anomalies with new member:
  \begin{description}
  \item[{\tt vct} (input)]: additional ensemble member;
  \item[{\tt varref} (input)]: new value of scalar reference ensemble;
  \item[{\tt idx} (input, optional)]: index of new member
                                      (present if weight and weightsum are not present);
  \item[{\tt weight} (input, optional)]: weight associated to each ensemble member
                                         (default=equal weights).
  \item[{\tt weightsum} (input/output, optional)]: current sum of weights;
  \item[{\tt mean} (input/output)]: ensemble mean;
  \item[{\tt meanref} (input/output)]: reference ensemble mean;
  \item[{\tt mproda} (input/output, optional)]: ensemble mean product of anomalies.
  \end{description}
\end{description}

\subsubsection*{MPI parallelization}

\subsubsection*{Computational cost}


\end{document}

