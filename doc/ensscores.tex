\documentclass[11pt]{article}

\usepackage[latin1]{inputenc}
\usepackage[a4paper,top=1in,left=1in,right=1in,bottom=1in]{geometry}
\usepackage[pdftex]{graphicx}
\usepackage{natbib}

\begin{document}

\pagestyle{empty}

\centerline{
\includegraphics[height=22mm]{Logos/logo_uga.png}
\hspace{5mm}
\includegraphics[height=22mm]{Logos/logo_cnrs.png}
\hfill
\includegraphics[height=22mm]{Logos/logo_ige.png}
}

\vspace{20mm}

\begin{center}

{\Huge\bf EnsScores}

\vspace{10mm}

{\Large\bf Ensemble Scores}

\vspace{10mm}

{\Large\bf User's guide}

\vspace{10mm}

{\large\bf Jean-Michel Brankart}

\vspace{5mm}
{\tt http://pp.ige-grenoble.fr/pageperso/brankarj/}

\vspace{5mm}
{\large Institut des G\'eosciences de l'Environnement}

\vspace{1mm}
{\large Universit\'e Grenoble Alpes, CNRS, France}

\end{center}

\vspace{20mm}
The purpose of EnsScores is to provide tools
to compute probabilistic scores of ensemble simulations.
The scores evaluates the reliability and the resolution of the simulation
by comparison to verification data.

The tools are provided as a library of modules,
which can be easily plugged in any existing software.
This library includes:

\begin{itemize}
\item the computation of the Continuous Rank Probability Score (CRPS);
\item the computation of the Reduced Centered Random variable (RCRV) score;
\item the computation of scores based on the relative entropy of user-defined events.
\end{itemize}

\clearpage

\pagestyle{plain}

\section{Description of the modules}

In this section,
the modules are described one by one,
giving for each of them:
the method that has been implemented,
the list of public variables and public routines
(with a description of input and output data),
the MPI parallelization, and
an estimation of the computational cost
as a function of the size of the problem.

\subsection{Module: {\tt\bf score\_crps}}

The purpose of this module is to compute the CRPS
of an ensemble simulation by comparison to verification data.

\subsubsection*{Method}

\subsubsection*{Public variables}

\begin{description}
\item[mpi\_comm\_score\_crps] MPI communicator to use (default=mpi\_comm\_world).
\item[crps\_missing\_value] missing value to use where no valid data is available (default=-9999.).
\end{description}

\subsubsection*{Public routines}

\begin{description}
\item[crps\_score:] compute CRPS score (with option to partition the data):
  \begin{description}
  \item[{\tt crps} (output)]: CRPS score for each subset of data;
  \item[{\tt reliability} (output)]: reliability part of CRPS;
  \item[{\tt resolution} (output)]: resolution part of CRPS;
  \item[{\tt ens} (input)]: ensemble to evaluate (model equivalent to verification data);
  \item[{\tt verif} (input)]: verification data;
  \item[{\tt partition} (input, optional)]: partition of verification data.
  \end{description}
\item[crps\_cumul:] accumulate data to prepare the final computation of the score
                    (for advanced use, if the full ensemble is only made progressively available).
\item[crps\_final:] compute final score from accumulated data
                    (for advanced use, if the full ensemble is only made progressively available).
\end{description}

\subsubsection*{MPI parallelization}

\subsubsection*{Computational cost}

\subsection{Module: {\tt\bf score\_rcrv}}

The purpose of this module is to compute the RCRV
of an ensemble simulation by comparison to verification data.

\subsubsection*{Method}

\subsubsection*{Public variables}

\begin{description}
\item[mpi\_comm\_score\_rcrv] MPI communicator to use (default=mpi\_comm\_world).
\item[rcrv\_missing\_value] missing value to use where no valid data is available (default=-9999.).
\item[rcrv\_with\_anamorphosis] use anamorphosis to compute reduced variable (default=.FALSE.).
\item[rcrv\_number\_of\_quantiles] number of quantiles to perform anamorphosis (default=11).
\end{description}

\subsubsection*{Public routines}

\begin{description}
\item[rcrv\_score:] compute RCRV score (with option to partition the data):
  \begin{description}
  \item[{\tt ens\_bias} (output)]: bias component of RCRV (should be 0);
  \item[{\tt ens\_spread} (output)]: spread component of RCRV (should be 1);
  \item[{\tt ens} (input)]: ensemble to evaluate (model equivalent to verification data);
  \item[{\tt verif} (input)]: verification data;
  \item[{\tt partition} (input, optional)]: partition of verification data.
  \end{description}
\item[rcrv\_cumul:] accumulate data to prepare the final computation of the score
                    (for advanced use, if the full ensemble is only made progressively available).
\end{description}

\subsubsection*{MPI parallelization}

\subsubsection*{Computational cost}

\subsection{Module: {\tt\bf score\_entropy}}

The purpose of this module is to compute
scores based on the relative entropy of user-defined events.

\subsubsection*{Method}

\subsubsection*{Public variables}

\begin{description}
\item[mpi\_comm\_score\_entropy] MPI communicator to use (default=mpi\_comm\_world).
\item[score\_entropy\_base] base to use in the computation of logarithms (default=2.).
\end{description}

\subsubsection*{Public routines}

\begin{description}
\item[events\_score:] compute entropy based score:
  \begin{description}
  \item[{\tt score} (output)]: ensemble score for each event;
  \item[{\tt ens} (input)]: ensemble to evaluate;
  \item[{\tt pref} (input)]: reference probability distribution for each event;
  \item[{\tt events\_outcome} (input)]: callback routine providing the outcome
                                        of the events for a given member.
  \end{description}
\item[events\_relative\_entropy:] compute relative entropy:
  \begin{description}
  \item[{\tt relative\_entropy} (output)]: relative entropy (with respect to reference distribution);
  \item[{\tt ens} (input)]: ensemble to evaluate;
  \item[{\tt pref} (input)]: reference probability distribution for each event;
  \item[{\tt events\_outcome} (input)]: callback routine providing the outcome
                                        of the events for a given member.
  \end{description}
\item[events\_cross\_entropy:] compute cross entropy:
  \begin{description}
  \item[{\tt cross\_entropy} (output)]: cross entropy (with reference distribution);
  \item[{\tt ens} (input)]: ensemble to evaluate;
  \item[{\tt pref} (input)]: reference probability distribution for each event;
  \item[{\tt events\_outcome} (input)]: callback routine providing the outcome
                                        of the events for a given member.
  \end{description}
\item[events\_entropy:] compute ensemble entropy:
  \begin{description}
  \item[{\tt entropy} (output)]: ensemble entropy;
  \item[{\tt number\_outcome} (input)]: number of possible outcomes for the events;
  \item[{\tt ens} (input)]: ensemble to evaluate;
  \item[{\tt events\_outcome} (input)]: callback routine providing the outcome
                                        of the events for a given member.
  \end{description}
\item[events\_probability:] compute events marginal probability distributions from the ensemble:
  \begin{description}
  \item[{\tt pens} (output)]: ensemble probability distribution for each event;
  \item[{\tt ens} (input)]: ensemble to evaluate;
  \item[{\tt events\_outcome} (input)]: callback routine providing the outcome
                                        of the events for a given member.
  \end{description}
\end{description}

\subsubsection*{MPI parallelization}

\subsubsection*{Computational cost}

\end{document}

